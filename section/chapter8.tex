\chapter{Perkenalan Generative Adversarial Network}
Menggunakan sumber buku \cite{ahirwar2019gan}. Dengan source code yang ada di github awangga. 
https://github.com/awangga/Generative-Adversarial-Networks-Projects
Tujuan pembelajaran pada pertemuan pertama antara lain:
\begin{enumerate}
\item
Mengerti konsep dasar Generative Adversarial Network
\item
Mengerti teknik Generator dan diskriminators
\item
Memahami penggunaan keras dan tensorflow
\end{enumerate}

Tugas dengan cara dikumpulkan dengan pull request ke github dengan menggunakan latex pada repo yang dibuat oleh asisten riset. Kode program menggunakan input listing ditaruh di folder src ekstensi .py dan dipanggil ke latex dengan input listings. \textbf{Tulisan dan kode tidak boleh plagiat}, menggunakan bahasa indonesia yang sesuai dengan gaya bahasa buku teks. Tidak menyertakan \textbf{pdf} kompilasi \textbf{diskon 50\%} nilainya.

\section{Teori}
Teori diambil dari buku referensi \cite{ahirwar2019gan}  chapter 1. 

\section{Soal Teori}
Praktek teori penunjang yang dikerjakan total nilai 100 sebagai nilai terpisah dari praktek pada modul ini(nilai 6,75 per nomor kecuali nomor terakhir 30 sisanya merupakan penanganan error, untuk hari pertama) :
\begin{enumerate}
\item
Jelaskan dengan ilustrasi gambar sendiri apa itu generator dengan perumpamaan anda sebagai mahasiswa sebagai generatornya.
\item
Jelaskan dengan ilustrasi gambar sendiri apa itu diskriminator dengan perumpamaan dosen anda sebagai diskriminatornya.
\item
Jelaskan dengan ilustrasi gambar sendiri bagaimana arsitektur generator dibuat
\item
Jelaskan dengan ilustrasi gambar sendiri bagaimana arsitektur diskriminator dibuat
\item
Jelaskan dengan ilustrasi gambar apa itu latent space.
\item
Jelaskan dengan ilustrasi gambar apa itu adversarial play
\item
Jelaskan dengan ilustrasi gambar apa itu Nash equilibrium
\item
Sebutkan dan jelaskan contoh-contoh implementasi dari GAN
\item
Berikan contoh dengan penjelasan kode program beserta gambar arsitektur untuk membuat generator(neural network) dengan sebuah input layer, tiga hidden layer(dense layer), dan satu output layer(reshape layer)
\item
Berikan contoh dengan ilustrasi dari arsitektur dikriminator dengan sebuath input layer, 3 buah hidden layer, dan satu output layer.
\item
Jelaskan bagaimana kaitan output dan input antara generator dan diskriminator tersebut. Jelaskan kenapa inputan dan outputan seperti itu.
\item 
Jelaskan apa perbedaan antara Kullback-Leibler divergence (KL divergence)/relative entropy, Jensen-Shannon(JS) divergence / information radius(iRaD) / total divergence to the average dalam mengukur kualitas dari model.
\item
Jelaskan apa itu fungsi objektif yang berfungsi untuk mengukur kesamaan antara gambar yang dibuat dengan yang asli.
\item
Jelaskan apa itu scoring algoritma selain mean square error atau cross entropy seperti The Inception Score dan The Frechet Inception distance.
\item
Jelaskan kelebihan dan kekurangan GAN
\end{enumerate}



\section{Praktek Program}
Tugas nilai terpisah dari teori maksimal 100. Praktekkan dengan menjalankan kode program nya dan jelaskan (diperlihatkan di video youtube) dengan menggunakan bahasa yang mudah dimengerti dan bebas plagiat dan wajib diambil dari layar komputer sendiri masing masing nomor di bawah ini(nilai 3 masing masing pada hari kedua). Buka kode program pada repo Generative-Adversarial-Networks-Projects pada github awangga. Buka folder Chapter02. Kita praktekkan 3D Generative Adversarial Networks (3D-GANs) .
\begin{enumerate}
\item
Jelaskan apa itu 3D convolutions
\item
Jelaskan dengan kode program arsitektur dari generator networknya, beserta penjelasan input dan output dari generator network.
\item
Jelaskan dengan kode program arsitektur dari diskriminator network, beserta penjelasan input dan outputnya.
\item 
Jelaskan proses training 3D-GANs
\item
Jelaskan bagaimana melakukan settingan awal chapter 02 untuk memenuhi semua kebutuhan sebelum melanjutkan ke tahapan persiapan data.
\item
Jelaskan tentang dataset yang digunakan, dari mulai tempat unduh, cara membuka dan melihat data. Sampai deskripsi dari isi dataset dengan detail penjelasan setiap folder/file yang membuat orang awam paham.
\item
Jelaskan apa itu voxel dengan ilustrasi dan bahasa paling awam
\item
Visualisasikan dataset tersebut dalam tampilan visual plot, jelaskan cara melakukan visualisasinya
\item
buka file run.py jelaskan perbaris kode pada fungsi untuk membuat generator yaitu build\_generator.
\item
jelaskan juga fungsi untuk membangun diskriminator pada fungsi build\_discriminator.
\item
jelaskan apa maksud dari kode program \_\_name\_\_ == '\_\_main\_\_'
\begin{lstlisting}[caption=Kode program utama,label={lst:8.1}]
if __name__ == '__main__':
\end{lstlisting}
\item
jelaskan secara detil perbaris dan per parameter apa arti dari kode program :
\begin{lstlisting}[caption=Setting Parameter,label={lst:8.2}]
    object_name = "chair"
    data_dir = "data/3DShapeNets/volumetric_data/" \
               "{}/30/train/*.mat".format(object_name)
    gen_learning_rate = 0.0025
    dis_learning_rate = 10e-5
    beta = 0.5
    batch_size = 1
    z_size = 200
    epochs = 10
\end{lstlisting}
\item
Jelaskan secara detil dari kode program pembuatan dan kompilasi arsitektur berikut :
\begin{lstlisting}[caption=Setting Parameter,label={lst:8.2}]
    gen_optimizer = Adam(lr=gen_learning_rate, beta_1=beta)
    dis_optimizer = Adam(lr=dis_learning_rate, beta_1=beta)

    discriminator = build_discriminator()
    discriminator.compile(loss='binary_crossentropy', optimizer=dis_optimizer)

    generator = build_generator()
    generator.compile(loss='binary_crossentropy', optimizer=gen_optimizer)
\end{lstlisting}
\item
Jelaskan secara detil kode program untuk membuat dan melakukan kompilasi model adversarial berikut:
\begin{lstlisting}[caption=Membuat dan Kompilasi Model Adversarial,label={lst:8.3}]
    discriminator.trainable = False

    input_layer = Input(shape=(1, 1, 1, z_size))
    generated_volumes = generator(input_layer)
    validity = discriminator(generated_volumes)
    adversarial_model = Model(inputs=[input_layer], outputs=[validity])
    adversarial_model.compile(loss='binary_crossentropy', optimizer=gen_optimizer)
\end{lstlisting}
\item
Jelaskan Ekstrak dan load data kursi dengan menggunakan fungsi getVoxelsFormat dan get3DImages yang digunakan pada kode program berikut :
\begin{lstlisting}[caption=Ekstraksi dan load dataset,label={lst:8.4}]
    print("Loading data...")
    volumes = get3DImages(data_dir=data_dir)
    volumes = volumes[..., np.newaxis].astype(np.float)
    print("Data loaded...")
\end{lstlisting}
\item
Jelaskan maksud dari kode program instansiasi TensorBoard yang menambahkan generator dan diskriminator pada program berikut:
\begin{lstlisting}[caption=Instansiasi tensorboard,label={lst:8.5}]
    tensorboard = TensorBoard(log_dir="logs/{}".format(time.time()))
    tensorboard.set_model(generator)
    tensorboard.set_model(discriminator)
\end{lstlisting}
\item
Jelaskan apa fungsi dari np reshape ones zeros pada kode program berikut dengan parameternya:
\begin{lstlisting}[caption=Pelabelan dataset,label={lst:8.6}]
    labels_real = np.reshape(np.ones((batch_size,)), (-1, 1, 1, 1, 1))
    labels_fake = np.reshape(np.zeros((batch_size,)), (-1, 1, 1, 1, 1))
\end{lstlisting}
\item
Jelaskan kenapa harus ada perulangan dalam meraih epoch. Dan jelaskan apa itu epoch terkait kode program berikut:
\begin{lstlisting}[caption=Setting Epoch,label={lst:8.7}]
        for epoch in range(epochs):
            print("Epoch:", epoch)

            gen_losses = []
            dis_losses = []
\end{lstlisting}
\item
Jelaskan apa itu batches dan kaitannya dengan kode program berikut, dan kenapa berada di dalam epoch:
\begin{lstlisting}[caption=Setting Batch,label={lst:8.8}]
            number_of_batches = int(volumes.shape[0] / batch_size)
            print("Number of batches:", number_of_batches)
            for index in range(number_of_batches):
                print("Batch:", index + 1)
\end{lstlisting}
\item
Berikut adalah kode program pengambilan gambar dan noise. Jelaskan apa fungsi np.random.normal serta astype, serta jelaskan apa arti parameter titik dua dan jelaskan isi dari z\_sample dan volumes\_batch:
\begin{lstlisting}[caption=Set real images dan vektor noise,label={lst:8.9}]
                z_sample = np.random.normal(0, 0.33, size=[batch_size, 1, 1, 1, z_size]).astype(np.float32)
                volumes_batch = volumes[index * batch_size:(index + 1) * batch_size, :, :, :]
\end{lstlisting}

\item
Berikut adalah kode program generator gambar palsu. Jelaskan apa fungsi generator.predict\_on\_batch, serta jelaskan apa arti parameter z\_sample:
\begin{lstlisting}[caption=Generator Gambar Palsu,label={lst:8.10}]
                # Next, generate volumes using the generate network
                gen_volumes = generator.predict_on_batch(z_sample)
\end{lstlisting}

\item
Berikut adalah kode program training diskriminator dengan gambar palsu dari generator dan gambar asil. Jelaskan apa maksudnya harus dilakukan training diskriminator secara demikian dan jelaskan apa isi loss\_fake dan loss\_real serta d\_loss dan fungsi train\_on\_batch.
\begin{lstlisting}[caption=Training Diskriminator,label={lst:8.11}]
                discriminator.trainable = True
                if index % 2 == 0:
                    loss_real = discriminator.train_on_batch(volumes_batch, labels_real)
                    loss_fake = discriminator.train_on_batch(gen_volumes, labels_fake)

                    d_loss = 0.5 * np.add(loss_real, loss_fake)
                    print("d_loss:{}".format(d_loss))

                else:
                    d_loss = 0.0
\end{lstlisting}

\item
Berikut adalah kode program training model adversarial yang terdapat generator dan diskriminator. Jelaskan apa bagaimana proses terbentuknya parameter z dan g\_loss:
\begin{lstlisting}[caption=Training adversarial model,label={lst:8.12}]
                z = np.random.normal(0, 0.33, size=[batch_size, 1, 1, 1, z_size]).astype(np.float32)
                g_loss = adversarial_model.train_on_batch(z, labels_real)
                print("g_loss:{}".format(g_loss))

                gen_losses.append(g_loss)
                dis_losses.append(d_loss)
\end{lstlisting}

\item
Berikut adalah kode program generate dan menyimpan gambar 3D setelah beberapa saat setiap epoch. Jelaskan mengapa ada perulangan dengan parameter tersebut, serta jelaskan arti setiap variabel beserta perlihatkan isinya dan artikan isinya :
\begin{lstlisting}[caption=Buat dan simpan gambar 3D,label={lst:8.13}]
                # Every 10th mini-batch, generate volumes and save them
                if index % 10 == 0:
                    z_sample2 = np.random.normal(0, 0.33, size=[batch_size, 1, 1, 1, z_size]).astype(np.float32)
                    generated_volumes = generator.predict(z_sample2, verbose=3)
                    for i, generated_volume in enumerate(generated_volumes[:5]):
                        voxels = np.squeeze(generated_volume)
                        voxels[voxels < 0.5] = 0.
                        voxels[voxels >= 0.5] = 1.
                        saveFromVoxels(voxels, "results/img_{}_{}_{}".format(epoch, index, i))
\end{lstlisting}

\item
Berikut adalah kode program menyimpan average losses setiap epoch. Jelaskan apa itu tensorboard dan setiap parameter yang digunakan pada kode program ini :
\begin{lstlisting}[caption=Simpan Average losses setiap epoch,label={lst:8.14}]
            # Write losses to Tensorboard
            write_log(tensorboard, 'g_loss', np.mean(gen_losses), epoch)
            write_log(tensorboard, 'd_loss', np.mean(dis_losses), epoch)
\end{lstlisting}

\item
Berikut adalah kode program menyimpan model. Jelaskan apa itu format h5 dan penjelasan dari kode program berikut :
\begin{lstlisting}[caption=Simpan model,label={lst:8.15}]
        generator.save_weights(os.path.join("models", "generator_weights.h5"))
        discriminator.save_weights(os.path.join("models", "discriminator_weights.h5"))
\end{lstlisting}

\item
Berikut adalah kode program testing model. Jelaskan dengan ilustrasi gambar dari mulai meload hingga membuat gambar 3D dengan menggunakan z\_sample, bisakah parameter z\_sample tersebut diubah2? :
\begin{lstlisting}[caption=Testing model,label={lst:8.16}]
        # Create models
        generator = build_generator()
        discriminator = build_discriminator()

        # Load model weights
        generator.load_weights(os.path.join("models", "generator_weights.h5"), True)
        discriminator.load_weights(os.path.join("models", "discriminator_weights.h5"), True)

        # Generate 3D models
        z_sample = np.random.normal(0, 1, size=[batch_size, 1, 1, 1, z_size]).astype(np.float32)
        generated_volumes = generator.predict(z_sample, verbose=3)

        for i, generated_volume in enumerate(generated_volumes[:2]):
            voxels = np.squeeze(generated_volume)
            voxels[voxels < 0.5] = 0.
            voxels[voxels >= 0.5] = 1.
            saveFromVoxels(voxels, "results/gen_{}".format(i))
\end{lstlisting}

\end{enumerate}

\section{Penanganan Error}
Dari praktek pemrograman yang dilakukan di modul ini, error yang kita dapatkan(hasil komputer sendiri) di dokumentasikan dan di selesaikan(nilai 5 per error yang ditangani. Untuk hari kedua):

\begin{enumerate}
	\item skrinsut error
	\item Tuliskan kode eror dan jenis errornya
	\item Solusi pemecahan masalah error tersebut
\end{enumerate}

\section{Presentasi Tugas}
Pada pertemuan ini, diadakan tiga penilaiain yaitu penilaian untuk tugas mingguan hari pertama dan hari kedua yang terpisah masing-masing dengan nilai maksimal 100. Kemudian dalam satu minggu kedepan maksimal sebelum waktu mata kuliah kecerdasan buatan. Ada presentasi kematerian dengan nilai presentasi yang terpisah dengan nilai maksimal 100. Jadi ada tiga komponen penilaiain pada pertemuan ini yaitu :
\begin{enumerate}
	\item tugas teori hari pertama(maks 100) modul ini
	\item tugas praktek hari kedua modul ini(maks 100)
	\item Presentasi tugas penjelasan CNN dan deep learning, Mempraktekkan kode python dan menjelaskan cara kerjanya(maks 100).
\end{enumerate}
Waktu presentasi pada jam kerja di IRC. Kriteria penilaian presentasi sangat sederhana, presenter akan ditanyai 20(10 praktek dan 10 teori) pertanyaan tentang pemahamannya menggunakan python untuk kecerdasan buatan dan teori konvolusi dan deep learning. jika presenter tidak bisa menjawab satu pertanyaan asisten maka nilai nol. Jika semua pertanyaan bisa dijawab maka nilai 100. Presentasi bisa diulang apabila gagal, sampai bisa mendapatkan nilai 100 dalam waktu satu minggu kedepan.


